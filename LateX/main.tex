\documentclass[12pt]{article}
\usepackage{graphicx} % for including figure 
\usepackage{geometry} % geometry package for mentioning margin length
\geometry{margin=3cm} %for setting margin; DON'T CHANGE THIS
\bibliographystyle{plain}
\usepackage[boxed]{algorithm2e} % the boxed option keeps the algorithm in a box in the pdf.
                                % You can also use the option ruled or algoruled for a different output.
                                % For more information about the algorithm2e package you can read the docuemntation 
                                % at https://mirror.easyname.at/ctan/macros/latex/contrib/algorithm2e/doc/algorithm2e.pdf

\SetKw{KwBy}{by} % macro to increase the counter in for loop 'by' <quantity>  

\title{Project documentation}
\author{Lorenzo Zanolin}

\begin{document}
\maketitle

\section{First section}
The aim of this project is the study of Yao's protocol \cite{yao} and an useful application of it.\\ More precisely, we will implement Secure multi-party computation; this field has the goal of creating methods for parties to jointly compute a function over their inputs while keeping those inputs private \cite{mpc}. In this project, the function we decided to implement is the \textit{8 bit sum}.


\section{Second section}

This is the content of the second section. This section contains the algorithms of my implementation


\begin{algorithm}[H]
\SetAlgoLined
\KwData{this text}
  \KwResult{how to write algorithm with \LaTeX2e }
 initialization\;
 \While{While condition}{
  instructions\;
  \eIf{condition}{
   instructions1\;
   instructions2\;
   }{
   instructions3\;
  }
}
\For{$i\gets0$ \KwTo $8$ \KwBy $2$}{
    Do something
    }
 \caption{How to write algorithms}\label{alg:myalgoname}
\end{algorithm}



\subsection{A subsection}
A subsection is created to organise some information togather within a section. This includes the example of how to include
a figure. This shows how we refer to algorithm \ref{alg:myalgoname}.

\bibliography{thud}
 

\end{document}