\documentclass[12pt]{article}
\usepackage{graphicx} % for including figure 
\usepackage{geometry} % geometry package for mentioning margin length
\usepackage{hyperref}
\usepackage{dirtree}
\usepackage{url}
\geometry{margin=3cm} %for setting margin; DON'T CHANGE THIS
\bibliographystyle{plain}
\usepackage[boxed]{algorithm2e} % the boxed option keeps the algorithm in a box in the pdf.
                                % You can also use the option ruled or algoruled for a different output.
                                % For more information about the algorithm2e package you can read the docuemntation 
                                % at https://mirror.easyname.at/ctan/macros/latex/contrib/algorithm2e/doc/algorithm2e.pdf

\SetKw{KwBy}{by} % macro to increase the counter in for loop 'by' <quantity>  

\title{Project documentation}
\author{Lorenzo Zanolin}

\begin{document}
\maketitle

\section{Introduction}
The aim of this project is the study of Yao's protocol \cite{yao} and an useful application of it. More precisely, we will implement Secure multi-party computation; this field has the goal of creating methods for parties to jointly compute a function over their inputs while keeping those inputs private \cite{mpc}. In this project, the function we decided to implement is the \textit{8 bit sum}.

\subsection{Description of the circuit}
We will present briefly the 8-bit sum circuit. There are two basic components in this construction:
\begin{itemize}
    \item \textit{\hyperref[half]{Half Adder}}: used to sum the right-most digit;
    \item \textit{\hyperref[full]{Full adder}}:  used to sum a generic digit in the number, ranging from position 1 to 8. It receives in input also carry of the previous sum.
\end{itemize}

\begin{figure}[!htb]
    \begin{minipage}{0.48\textwidth}
        \centering
        \includegraphics[width=.7\linewidth]{../src/images/Half_adder.png}
        \caption{Half Adder}\label{half}
    \end{minipage}
    \hfill
    \begin{minipage}{0.48\textwidth}
        \centering
        \includegraphics[width=.8\linewidth]{../src/images/Full-adder.png}
        \caption{Full Adder}\label{full}
    \end{minipage}
\end{figure}

\footnote{\ref{half} was taken over \url{https://upload.wikimedia.org/wikipedia/commons/1/14/Half-adder.svg}\\ 
\ref{full} was taken over \url{https://upload.wikimedia.org/wikipedia/commons/a/a9/Full-adder.svg}.}

We then proceede creating the circuit by wiring 7 full adders and an half adder together, as represented in Figure \ref{circuit}.

\begin{figure}[]
    \centering
    \includegraphics[width=1\linewidth]{../src/images/Circuit.png}
    \caption{Full Adder}\label{circuit}
\end{figure}

\subsection{Implementation}
The project will be developed using \textit{Python 3.9.10} and we will use functions provided in the GitHub repo \url{https://github.com/ojroques/garbled-circuit}.
\subsubsection{Project structure}
The project is structured as follows:
\dirtree{%
.1 src/..
.2 Makefile.
.2 images \hspace{5cm} \begin{minipage}[t]{10cm}
    This directory contains the images used{.}
    \end{minipage}.
.3 8-bit\_full\_adder.png.
.3 Circuit.png.
.3 Half\_adder.png.
.3 Full-adder.png.
.2 circuits \hspace{4.5cm} \begin{minipage}[t]{10cm}
    This directory contains the circuit used{.}
    \end{minipage}.
.3 add.json.
.2 code \hspace{4.5cm} \begin{minipage}[t]{10cm}
    This directory contains the code used{.}
.3 util.py.
.3 yao.py.
.3 ot.py.
.3 requirements.py.
.3 main.py.
.2 sets.
.3 alice.txt.
.3 bob.txt.
}


\bibliography{thud}
 

\end{document}